% ====================================================================
% Abstract 
% ====================================================================
\noindent
Ruhland, Felipe. \textbf{Estudo sobre container linux para execução de aplicações web.}
Florianópolis, 2016. \pageref{nropaginas}f. Trabalho de Conclusão de Curso Superior de Tecnologia em
Análise e Desenvolvimento de Sistemas - Curso Análise e Desenvolvimento de Sistemas. Faculdade de Tecnologia do
SENAI, Florianópolis, 2016.

\vspace{1cm}
\begin{resumo}[\textbf{ABSTRACT}]
 \begin{otherlanguage*}{english}

Today, millions of applications are running in the cloud, on smartphones, tables or even on the web. Practically every application has an engine running on the cloud that labor demand for specialized to maintenance, development and monitoring of these applications. The cloud is occupied by virtual machines that are available to developers runs their applications. It is likely that this scenario change because the linux containers as taking space in production environments and must fight with virtual machines for most runs in the cloud. The Docker is a tool to seek to facilitate the creation of Linux containers to improve the way applications are developed, tested and distributed. Another factor to adopt the use of Docker is because of the computational resource savings that can be achieved. With the use of a microserviços architecture, it is possible to use resources more intelligently and appropriately machine, with the possibility of expanding the processes that need more resources. Thus, it is understood that this issue is very current and important for the future of cloud software and that is likely to be the most cases in the near future.
   \vspace{\onelineskip}
 
   \noindent 
   \textbf{Key-words}: Docker. Container. Web Application.
 \end{otherlanguage*}
\end{resumo}

