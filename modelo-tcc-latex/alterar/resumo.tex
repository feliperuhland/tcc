% ====================================================================
% Resumo 
% ====================================================================

\noindent
Ruhland, Felipe. \textbf{Estudo sobre container linux para execução de aplicações web.}
Florianópolis, 2016. \pageref{nropaginas}f. Trabalho de Conclusão de Curso Superior de Tecnologia em
Análise e Desenvolvimento de Sistemas - Curso Análise e Desenvolvimento de Sistemas. Faculdade de Tecnologia do
SENAI, Florianópolis, 2016.

\vspace{1cm}
\setlength{\absparsep}{18pt} % ajusta o espaçamento dos parágrafos do resumo
\begin{resumo}
Atualmente, milhões de aplicações estão rodando na nuvem, estejam elas em smartphones, tables ou mesmo na web. Praticamente todo aplicativo tem um motor rodando na núvem, que demanda de mão de obra especializada para a manutenção, evolução e monitoramento dessas aplicações. A nuvem é praticamente ocupada por máquinas virtuais que são disponibilizadas para terceiros rodarem suas aplicações. É provável que este cenário mude, pois os containers linux estão tomando espaço nos ambientes de produção e devem brigar com as máquinas virtuais pela maioria das execuções na nuvem. O Docker é uma ferramenta que buscar facilitar a criação de containers linux para melhorar a forma que as aplicações são desenvolvidas, testadas e distribuídas. Outro fator para adotar o uso do Docker é em razão da economia de recurso computacional que é possível atingir. Com a utilização de uma arquitetura de microserviços, é possível utilizar os recursos da máquina de maneira mais inteligente e apropriada, com a possibilidade de ampliar as processos que precisem de mais recursos. Desta forma, entende-se que este assunto é muito atual e importante para o futuro do software em nuvem e que é provável que seja a maioria dos casos num futuro pŕoximo.

 \textbf{Palavras-chave}: Docker. Container. Aplicação web.
\end{resumo}
