% ====================================================================
% Resumo 
% ====================================================================

\noindent
Ruhland, Felipe. \textbf{Estudo sobre container linux para execução de aplicações web.}
Florianópolis, 2016. \pageref{nropaginas}f. Trabalho de Conclusão de Curso Superior de Tecnologia em
Análise e Desenvolvimento de Sistemas - Curso Análise e Desenvolvimento de Sistemas. Faculdade de Tecnologia do
SENAI, Florianópolis, 2016.

\vspace{1cm}
\setlength{\absparsep}{18pt} % ajusta o espaçamento dos parágrafos do resumo
\begin{resumo}
 Segundo a NBR6028:2003, o resumo deve ressaltar o
 objetivo, o método, os resultados e as conclusões do documento. A ordem e a extensão
 destes itens dependem do tipo de resumo (informativo ou indicativo) e do
 tratamento que cada item recebe no documento original. O resumo deve ser
 precedido da referência do documento, com exceção do resumo inserido no
 próprio documento. (\ldots) As palavras-chave devem figurar logo abaixo do
 resumo, antecedidas da expressão Palavras-chave:, separadas entre si por
 ponto e finalizadas também por ponto.

 \textbf{Palavras-chave}: Latex. Abntex. Editoração de texto.
\end{resumo}
