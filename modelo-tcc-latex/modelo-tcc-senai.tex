%% abtex2-modelo-trabalho-academico.tex, v-1.9 laurocesar
%% Copyright 2012-2013 by abnTeX2 group at http://abntex2.googlecode.com/ 
%%
%% This work may be distributed and/or modified under the
%% conditions of the LaTeX Project Public License, either version 1.3
%% of this license or (at your option) any later version.
%% The latest version of this license is in
%%   http://www.latex-project.org/lppl.txt
%% and version 1.3 or later is part of all distributions of LaTeX
%% version 2005/12/01 or later.
%%
%% This work has the LPPL maintenance status `maintained'.
%% 
%% The Current Maintainer of this work is the abnTeX2 team, led
%% by Lauro César Araujo. Further information are available on 
%% http://abntex2.googlecode.com/
%%
%% This work consists of the files abntex2-modelo-trabalho-academico.tex,
%% abntex2-modelo-include-comandos and abntex2-modelo-references.bib
%%

% ------------------------------------------------------------------------
% ------------------------------------------------------------------------
% abnTeX2: Modelo de Trabalho Academico (tese de doutorado, dissertacao de
% mestrado e trabalhos monograficos em geral) em conformidade com 
% ABNT NBR 14724:2011: Informacao e documentacao - Trabalhos academicos -
% Apresentacao
% ------------------------------------------------------------------------
% ------------------------------------------------------------------------

\documentclass[
	% -- opções da classe memoir --
	12pt,				% tamanho da fonte
	openright,			% capítulos começam em pág ímpar (insere página vazia caso preciso)
	oneside,			% para impressão em verso e anverso. Oposto a oneside
	a4paper,			% tamanho do papel. 
	% -- opções da classe abntex2 --
	chapter=TITLE,		% títulos de capítulos convertidos em letras maiúsculas
	section=TITLE,		% títulos de seções convertidos em letras maiúsculas
	%subsection=TITLE,	% títulos de subseções convertidos em letras maiúsculas
	%subsubsection=TITLE,% títulos de subsubseções convertidos em letras maiúsculas
	% -- opções do pacote babel --
	english,			% idioma adicional para hifenização
	french,				% idioma adicional para hifenização
	spanish,			% idioma adicional para hifenização
	brazil				% o último idioma é o principal do documento
	]{abntex2}

% ---
% PACOTES
% ---

% ---
% Pacotes fundamentais 
% ---
%\usepackage{lmodern}			% Usa a fonte Latin Modern			
\usepackage{times}			% Usa a fonte Times
\usepackage[T1]{fontenc}		% Selecao de codigos de fonte.
\usepackage[utf8]{inputenc}		% Codificacao do documento (conversão automática dos acentos)
\usepackage{lastpage}			% Usado pela Ficha catalográfica
\usepackage{indentfirst}		% Indenta o primeiro parágrafo de cada seção.
\usepackage{color}			% Controle das cores
\usepackage{graphicx}			% Inclusão de gráficos
\usepackage{microtype} 			% para melhorias de justificação
\usepackage{listings}			% para inserir código fonte

% ---
% Pacotes de citações
% ---
\usepackage[brazilian,hyperpageref]{backref}	 % Paginas com as citações na bibl
\usepackage[alf]{abntex2cite}	% Citações padrão ABNT
\usepackage{titlesec}

% --- 
% CONFIGURAÇÕES DE PACOTES
% --- 

% altera o espacamento depois do número de cada secao, subsecao, etc
\titleformat{\section}
  {\normalfont\normalsize}{}{0pt}{\thesection\space}
\titleformat{\subsection}
  {\normalfont\normalsize\bfseries}{}{0pt}{\thesubsection\space}
\titleformat{\subsubsection}
  {\normalfont\normalsize}{}{0pt}{\thesubsubsection\space}
\titleformat{\paragraph}
  {\normalfont\normalsize\itshape}{}{0pt}{\theparagraph\space}

% ---
% Configurações do pacote backref
% Usado sem a opção hyperpageref de backref
\renewcommand{\backrefpagesname}{Citado na(s) página(s):~}
% Texto padrão antes do número das páginas
\renewcommand{\backref}{}
% Define os textos da citação
\renewcommand*{\backrefalt}[4]{
	\ifcase #1 %
		Nenhuma citação no texto.%
	\or
		Citado na página #2.%
	\else
		Citado #1 vezes nas páginas #2.%
	\fi}%
% ---

% ---
% **************************************************
% Informações que devem ser alteradas
% **************************************************
\titulo{\uppercase{Estudo sobre container linux para execução de aplicações web}}
\autor{\uppercase{Felipe Mendonça Ruhland}}
\orientador{Paulo Bueno}
\orientadortcc{Prof. \imprimirorientador, Dr. (SENAI/SC)}
\coordenador{Prof. Bobiquins Estevão de Mello, Me. (SENAI/SC)}
\coordenadortcc{Profa. Jaqueline Voltolini de Almeida, Me. (SENAI/SC)}
\examinador{Prof. Fulado de tal, Me. (SENAI/SC)}
\preambulo{Trabalho de Conclusão de Curso apresentado à Banca Examinadora do Curso Superior de Tecnologia em Redes de Computadores da Faculdade de Tecnologia do SENAI Florianópolis como requisito parcial para obtenção do Grau de Tecnólogo em Análise e Desenvolvimento de Sistemas.}
\proforientador{Professor Orientador: \imprimirorientador.}
\datadefesa{\uppercase{?? de julho de 2016}}
\local{Florianópolis/SC}
\data{2016}
% **************************************************

\instituicao{%
  SERVIÇO NACIONAL DE APRENDIZAGEM INDUSTRIAL 
  \par
  FACULDADE DE TECNOLOGIA SENAI/SC FLORIANÓPOLIS
  \par
  CURSO SUPERIOR DE TECNOLOGIA EM REDES DE COMPUTADORES}
\tipotrabalho{Trabalho de Conclusão de Curso}

% alterando o aspecto da cor azul
\definecolor{blue}{RGB}{41,5,195}

% informações do PDF
\makeatletter
\hypersetup{
     	%pagebackref=true,
		pdftitle={\@title}, 
		pdfauthor={\@author},
    	pdfsubject={\imprimirpreambulo},
	    pdfcreator={LaTeX with abnTeX2},
		pdfkeywords={abnt}{latex}{abntex}{abntex2}{trabalho acadêmico}, 
		colorlinks=true,      	% false: boxed links; true: colored links
		linkcolor=black,	% color of internal links
		citecolor=black,        % color of links to bibliography
		filecolor=magenta,      % color of file links
		urlcolor=black,
		bookmarksdepth=4
}
\makeatother
% --- 

% --- 
% Espaçamentos entre linhas e parágrafos 
% --- 

% O tamanho do parágrafo é dado por:
\setlength{\parindent}{1.3cm}

% Controle do espaçamento entre um parágrafo e outro:
\setlength{\parskip}{0.2cm}  % tente também \onelineskip

%\titlespacing\section{0pt}{12pt plus 4pt minus 2pt}{-6pt plus 2pt minus 2pt}

% ---
% compila o indice
% ---
\makeindex
% ---

% ----
% Início do documento
% ----
\begin{document}

% Retira espaço extra obsoleto entre as frases.
\frenchspacing 

% ----------------------------------------------------------
% ELEMENTOS PRÉ-TEXTUAIS
% ----------------------------------------------------------

\imprimircapa
\imprimirfolhaderosto*

\begin{folhadeaprovacao}

  \begin{center}
    {\ABNTEXchapterfont\bfseries\normalsize\imprimirautor}

    \vspace*{\fill}\vspace*{\fill}
    \begin{center}
      \ABNTEXchapterfont\bfseries\normalsize\imprimirtitulo
    \end{center}
    \vspace*{\fill}
        \imprimirpreambulo
    \vspace*{\fill}
        
   APROVADA PELA {\bfseries{COMISSÃO EXAMINADORA}}
   \par
   EM FLORIANÓPOLIS, \bfseries{\imprimirdatadefesa}
   \end{center}

   \assinatura{\imprimircoordenador \\ Coordenador do Curso} 
   \assinatura{\imprimircoordenadortcc \\ Coordenador de TCC} 
   \assinatura{\imprimirorientadortcc \\ Orientador} 
   \assinatura{\imprimirexaminador \\ Examinador}
   \begin{center}
    \vspace*{1cm}
  \end{center}
\end{folhadeaprovacao}

% Arquivos que devem ser alterados
% ====================================================================
% Dedicatória 
% ====================================================================
\begin{dedicatoria}
Dedico este trabalho ao meu avô Huri Gomes Mendonça, \textit{in memoriam}, que sempre me apoiou nos estudos mesmo sem entender o contexto do curso.

\end{dedicatoria}

% ====================================================================
% Agradecimentos 
% ====================================================================
\begin{agradecimentos}
Agradeço à minha família, minha noiva, meus amigos e colegas de trabalho pela ajuda e pelo apoio de sempre.
\end{agradecimentos}

% ====================================================================
% Epigrafe 
% ====================================================================
\begin{epigrafe}
    \vspace*{\fill}
	\begin{flushright}
		\textit{``Talk is cheap. Show me the code''} \\
		(LINUS TORVALDS)
	\end{flushright}
\end{epigrafe}


% ====================================================================
% Resumo 
% ====================================================================

\noindent
SOBRENOME, Nome. \textbf{Título do trabalho.}
Florianópolis, 2013. \pageref{nropaginas}f. Trabalho de Conclusão de Curso Superior de Tecnologia em
Redes de Computadores - Curso Redes de Computadores. Faculdade de Tecnologia do
SENAI, Florianópolis, 2013.

\vspace{1cm}
\setlength{\absparsep}{18pt} % ajusta o espaçamento dos parágrafos do resumo
\begin{resumo}
 Segundo a NBR6028:2003, o resumo deve ressaltar o
 objetivo, o método, os resultados e as conclusões do documento. A ordem e a extensão
 destes itens dependem do tipo de resumo (informativo ou indicativo) e do
 tratamento que cada item recebe no documento original. O resumo deve ser
 precedido da referência do documento, com exceção do resumo inserido no
 próprio documento. (\ldots) As palavras-chave devem figurar logo abaixo do
 resumo, antecedidas da expressão Palavras-chave:, separadas entre si por
 ponto e finalizadas também por ponto.

 \textbf{Palavras-chave}: Latex. Abntex. Editoração de texto.
\end{resumo}

% ====================================================================
% Abstract 
% ====================================================================
\noindent
SOBRENOME, Nome. \textbf{Título do trabalho.}
Florianópolis, 2013. 89f. Trabalho de Conclusão de Curso Superior de Tecnologia em
Redes de Computadores - Curso Redes de Computadores. Faculdade de Tecnologia do
SENAI, Florianópolis, 2013.

\vspace{1cm}
\begin{resumo}[\textbf{ABSTRACT}]
 \begin{otherlanguage*}{english}
   This is the english abstract.

   \vspace{\onelineskip}
 
   \noindent 
   \textbf{Key-words}: Latex. Abntex. Text editoration.
 \end{otherlanguage*}
\end{resumo}



% lista de figuras
\pdfbookmark[0]{\listfigurename}{lof}
\listoffigures*
\clearpage

% inserir lista de tabelas
\pdfbookmark[0]{\listtablename}{lot}
\listoftables*
\clearpage

% Arquivos que devem ser alterados
% ====================================================================
% Siglas 
% ====================================================================

\begin{siglas}
  \item[CGroups] Control Groups
  \item[LXC] Linux Container
  \item[VPS] Servidores privados virtuais
  \item[TI] Tecnologia da Informação
\end{siglas}


\include{alterar/simbolos}

% inserir o sumario
\pdfbookmark[0]{\contentsname}{toc}
\tableofcontents*
\clearpage

% ----------------------------------------------------------
% ELEMENTOS TEXTUAIS
% ----------------------------------------------------------
\textual

% PARTE - preparação da pesquisa
% ----------------------------------------------------------
%\part{Preparação da pesquisa}


% Informações que devem ser alteradas
% **************************************************
% ---
% Introdução
% ---
\chapter{INTRODUÇÃO}

Mais de dois terços das aplicações web rodam em ambiente unix, segundo
\url{http://w3techs.com/technologies/overview/operating\_system/all}. Os sistemas são executados em servidores dedicados ou virtualizados. Os servidores dedicados são a maneira mais natural de servir uma aplicação web. São máquinas físicas, normalmente em datacerters, com um sistema operacional linux e com todo o hardware disponível.
Entretanto, executar uma aplicação web em um servidor dedicado acaba por desperdiçar muitos recursos do mesmo. Para combater este desperdício, trabalha-se há décadas para aperfeiçoar um servidor que consiga evitar o disperdício, com a divisão dos recursos.

Nos últimos anos, usou-se muito as máquinas virtuais para aproveitar melhor os recursos dos servidores. Elas funcionam como novas máquinas dentro da máquina física e podem ser incluídas na rede como se fossem uma máquina física. Desta maneira, um servidor dedicado passa a ser multiplas máquinas, com seus próprios recursos e totalmente isoladas, que traz ainda mais segurança para os administradores de sistemas.

Com esta estratégia, os VPS (Servidores privados virtuais) tornaram muito mais acessíveis ao público em geral, pois era possível contratar um pequeno servidor virtualizado e ter total controle das configurações desde servidor. Isso colaborou com pequenos empreendedores que puderam expor seu trabalho de maneira mais economica e, com isso, abrir um leque de possibilidades para novas empresas.

A máquina virtual, por padrão, funciona como uma máquina totalmente nova. Ou seja, ela possui sistema operacional próprio, permissões individuais e recursos compartilhados com da máquina anfitriã. É possível instalar diversas versões do kernel, por exemplo, sem que uma interfira na outra. Contudo, observa-se que para cada máquina virtual criada num servidor dedicado existe uma sobrecarga do sistema operacional, pois além do sistema instalado na máquina anfitriã, cada máquina virtual possui seu sistema individual. Sabe-se, também, que o sistema operacional necessida de uma série de recursos para o bom funcionamento, de maneira que os recursos não podem ser muito inferior para não ocorrer problemas. Em paralaleo às máquinas virtuais, existe uma outra alternativa, mais leve, chamada de container.

Conteiners linux existem com a finalidade de isolar ambientes dentro de um sistema operacional linux. Eles não são máquinas virtuais, mas também conseguem restringir acesso, definir recursos próprios, mas não sobrepõe o sistema operacional. Ele usa o sistema da máquina anfitriã e, com isso, consegui utilizar menos recursos que a máquina virtual. Em razão dessas vantagens, muitos estudam para deixar a criação e manutenção desses containers uma tarefa mais fácil para os profissionais de ti. O caso mais conhecido nos últimos anos é da ferramente Docker, que descomplicou a maneira de criar e gerir containers dentro de um sistema operacional. Pode-se dizer que o projeto é um sucesso, pois recebe mais utilizadores a cada dia e já possui investimentos na casa dos milhões de dólares.

Fala-se muito na arquitetura de microserviço, que traz vantagens para o desenvolvedor, por ter um escopo reduzido, facilita a criação e execução de testes, deploy e inúmeros outros fatores. Essa nova abordagem, traz consigo a ideia de executar o microserviço em um container para simplificar a infraestrutura, de modo a facilitar a escalabilidade da aplicação e a sua manutenção. Com este pensamento, pode-se criar milhares de containers com a mesma aplicação, em ambientes isolados, independentes e seguros.

Este projeto tem por objetivo fazer um estudo sobre a tecnologia de containers linux para a execução de aplicações web, em especial o Docker.


\section{JUSTIFICATIVA}

Em razão do Docker, tem-se discutido muito o assunto de containers para execução de aplicações web, com muitos olhares positivos e muita adesão. Acredita-se que é um assunto muito relevante, pois já chamou a atenção dos gigantes da TI, como Google, Red Hat e Microsoft. Já existem inúmeros serviços que utilizam a ideia de container para execução de aplicações e muitas empresas já confiam neste conceito. Inclusive, a grande justificativa que se traz, a princípio, é a vantagem de ter uma aplicação que roda da mesma forma em desenvolvimento, testes, integração e produção. Não existe mais a desculpa que o ambiente não estava identico, pois agora, o ambiente é reduzido a um container linux.

Conforme descrito, os estudos focam a solução Docker, por ser o grande responsável pelo assunto atualmente e por ter chamado tanta atenção dos profissionais de TI, uma vez que o Docker é bem quisto por todas as etapas do desenvolvimento de software.

\section{OBJETIVOS}

Nesta seção são apresentados os objetivos do presente trabalho.

\subsection{Objetivo geral}

Estudo da solução de container linux para a execução de aplicações web.

\subsection{Objetivos específicos}

\begin{enumerate}
	\item{Introdução ao Docker}
	\item{Diferenças do servidor dedicado, virtualizado e container}
	\item{Estudo do funcionamento do Docker}
	\item{Vantagens em executar aplicações em container}
	\item{Exemplos de utilização}
\end{enumerate}

\section{METODOLOGIA}

Dizer qual metodologia de trabalho será usada.

\section{ESTRUTURA DO TRABALHO}

Explicar como o trabalho está estruturado.

% ---
% Capitulo de revisão de literatura
% ---
\chapter{REVISÃO DA LITERATURA}\label{cap-revisao}

O processo de desenvolvimento de software é um trabalho complexo e depende de muitas etapas até a conclusão \url{http://www.devmedia.com.br/atividades-basicas-ao-processo-de-desenvolvimento-de-software/5413}.

\section{Introdução ao Docker}

\subsection{Ambiente de Desenvolvimento e Homologação}

Um grande desafio que os desenvolvedores enfrentam é a configuração do ambiente de desenvolvimento, pois o este ambiente acaba por ser simplificado em comparação ao ambiente de produção. É muito comum o software correr de maneira estável em desenvolvimento, mas não obter o mesmo resultado em produção. Seja banco de dados em memória, configurações padrão ou uma arquitetura mais modesta, os ambiente de desenvolvimento são mais simples para garantir que o desenvolvedor não perca tempo com configurações e ajustes e passa a focar no objetivo principal que é a implementação de funcionalidades.
Grandes empresas dispõe de um ambiente de homologação que precede o ambiente de produção. Isto ocorre, demaneira geral, para previnir erros e desacertos entre ambientes. Este procedimento sempre foi muito bem aceito \url{http://www.profissionaisti.com.br/2013/06/a-importancia-de-um-ambiente-de-homologacao/}, mas será que o ambiente de homologação é realmente necessário?

O ambiente de homologação exige, também, uma instalação completa. Exige instalação do software, de suas dependências, alterações de banco de dados e configurações de ambiente. Mesmo que seja utilizada uma ferramenta de automatização para a instalação, um ambiente de homologação exige atenção e manutenção para operar normalmente.

\subsection{Instalação do Software}

Após a implementação de uma ou mais funcionalidade, uma nova versão é gerada e deve ser lançada no ambiente de homologação ou produção. Normalmente este processo é arduo e depende de uma equipe especializada para concluir esta etapa. A equipe de desenvolvimento deve passar detalhes da implementação e as peculiaridades da versão. Qualquer erro ocorrido deve ser reiniciado o processo de instalação e pode provocar consequências irreversíveis, como corrupção de dados. Após a instalação, pode ocorrer a necessidade de ampliar os servidores para servir um tráfego maior de usuários e isso acarreta na instalação em novas máquinas. Mas, é claro, que essa instalação não acontece há tempo de suprir a necessidade de tráfego.

\subsection{Escalabilidade}

O aumento repentino de tráfego em aplicações web demandam muita estratégia da equipe de infraestrutura para que os usuários não sintam lentidão ou não recebam resposta do servior. Essa estratégia deve conter um rápido ataque para suprir esta necessidade e não pode contar com a instalação de novas máquinas físicas, pois não haveria tempo hábil para tal. Nos dias de hoje, uma empresa que oferece serviços como produto não pode deixar de prestar o serviço sob pena de ter o descrédito pelos usuários e resultar no término dos negócios. Portanto, esses casos exigem que a equipe de infraestrutura possua condições de ampliar o servidor web.

\subsection{Entregabilidade}

A dificuldade de instalar um software normalmente é medida pela frequência que ela ocorre. Quando ocorre frequentemente, deixa a instalação simples e rápida. Se ocorre com pouca frequência, deixa a instalação complexa e demorada. A forma tradicional de instalação normalmente é mais dolorosa e pouco frequente, em razão de diversos fatores como dependências, scripts de migração de banco de dados e etc. Assim, a entregabilidade é prejudicada, pois uma nova funcionalidade leva tempo para ser instalada em produção.

\subsection{Segurança}

Nos servidores físicos são comuns a execução de várias aplicações web na mesma máquina para utilizar o máximo de recurso possível. Entretanto, isso pode ser muito perigoso, pois uma vulnerabilidade em uma aplicação pode dar acesso à maquina que rodam as outras. Da mesma forma, um vazamento de memória pode impedir o bom funcionamento das demais aplicações e acabar derrubando as demais.

\subsection{O que é Docker?}

Docker é uma ferramenta de linha de comando, que é executada em plano de fundo, e promove um servidor remoto para simplificar a experiência de instalar, executar, publicar e remover software, segundo \citeonline[p.6, tradução livre]{inaction2016}. Possibilita que um software seja posto em um container, junto com suas dependências, em uma unidade padrão de desenvolvimento de software, conforme \citeonline[tradução livre]{sitedocker-whatdocker}. Desta forma, pode-se garantir que o software sempre vai ser comportar da mesma maneira, independente do ambiente que for executado.

Em março de 2013, o Docker foi lançado como um projeto de código aberto pela dotCloud, empresa que gerencia um serviço de plataforma na nuvem. Com o intuito de oferecer um serviço melhor e mais competitivo, o fundador e CEO da dotCloud desde 2010, Solomon Hykes, criou o Docker. Em pouco tempo, a ferramenta caiu nos braços da comunidade de desenvolvedores, inclusive grandes empresas de tecnologia como Red Hat, IBM, Google e Cisco, que ajudaram no desenvolvimento do produto, de acordo com \citeonline[tradução livre]{techtarget}.

O ano de 2014 foi um ano muito especial para o Docker, pois foi o ano que recebeu grandes investimentos da Greylock Parteners e Sequoia Capital e passou a ter um valor de mercado em US\$ 400M (Quatrocentos milhões de dolares). Em 2015, não foi muito diferente, houveram mais duas rodadas de investimentos que totalizou US\$ 180M (Cento e oitenta milhões de dolares), conforme  \citeonline[tradução livre]{crunchbase-docker}. 

Entre as características marcantes da ferramenta pode-se considerar a levesa, pois compartilha recursos do sistema operacional; abertura, pois pode ser executado na grande maioria dos servidores linux e com versões instáveis para OSX e Windows; seguro, pois o container promove mais uma barreira protetora para o ambiente, segundo \citeonline[tradução livre]{sitedocker-whatdocker}.

\section{Diferenças entre servidor dedicado, virtualizado e containerizado}



% ---
% Procedimentos metodológicos
% ---
\chapter{PROCEDIMENTOS METODOLÓGICOS}

Descrevem-se, neste capítulo, os procedimentos metodológicos que nortearam a pesquisa.

% ---
% Resultados
% ---
\chapter{RESULTADOS E DISCUSSÕES}

Neste capítulo são apresentados os resultados da pesquisa descrita no capítulo \ref{cap-revisao}.

% ---
% Conclusão
% ---
\chapter{CONCLUSÃO}

As conclusão do trabalho são apresentadas aqui.

% **************************************************

% ----------------------------------------------------------
% ELEMENTOS PÓS-TEXTUAIS
% ----------------------------------------------------------
% \postextual

% ----------------------------------------------------------
% Referências bibliográficas
% Arquivos que devem ser alterados
\bibliography{alterar/referencias}

% ----------------------------------------------------------
% Glossário
% ----------------------------------------------------------
%
% Consulte o manual da classe abntex2 para orientações sobre o glossário.
%
%\glossary

% ----------------------------------------------------------
% Apêndices
% ----------------------------------------------------------

% Informações que devem ser alteradas
% **************************************************
% ---
% Inicia os apêndices
% ---
\begin{apendicesenv}

\chapter{Código fonte}
Código de minha autoria. O apêndice é opcional ao TCC e deve ser elaborado pelo próprio autor. Destina-se a complementar as ideias, sem prejuízo do tema do trabalho. Segue um exemplo:

\scriptsize
\begin{lstlisting}
#include <stdio.h>

int main() {
  printf("Ola mundo !\n");
  return 0;
}
\end{lstlisting}

\end{apendicesenv}

% ----------------------------------------------------------
% Anexos
% ----------------------------------------------------------
\begin{anexosenv}

\chapter{Pesquisa IBGE}
O anexo é opcional ao TCC e são informações não elaboradas pelo próprio autor, mas que tem como objetivo complementar as ideias, sem prejuízo do tema do relatório.

\end{anexosenv}

% Etiqueta para auxiliar contagem do numero de paginas do texto e dos elementos pos-textuais
\label{nropaginas}

% **************************************************

%---------------------------------------------------------------------
% INDICE REMISSIVO
%---------------------------------------------------------------------
\printindex

\end{document}
