%% abtex2-modelo-trabalho-academico.tex, v-1.9 laurocesar
%% Copyright 2012-2013 by abnTeX2 group at http://abntex2.googlecode.com/ 
%%
%% This work may be distributed and/or modified under the
%% conditions of the LaTeX Project Public License, either version 1.3
%% of this license or (at your option) any later version.
%% The latest version of this license is in
%%   http://www.latex-project.org/lppl.txt
%% and version 1.3 or later is part of all distributions of LaTeX
%% version 2005/12/01 or later.
%%
%% This work has the LPPL maintenance status `maintained'.
%% 
%% The Current Maintainer of this work is the abnTeX2 team, led
%% by Lauro César Araujo. Further information are available on 
%% http://abntex2.googlecode.com/
%%
%% This work consists of the files abntex2-modelo-trabalho-academico.tex,
%% abntex2-modelo-include-comandos and abntex2-modelo-references.bib
%%

% ------------------------------------------------------------------------
% ------------------------------------------------------------------------
% abnTeX2: Modelo de Trabalho Academico (tese de doutorado, dissertacao de
% mestrado e trabalhos monograficos em geral) em conformidade com 
% ABNT NBR 14724:2011: Informacao e documentacao - Trabalhos academicos -
% Apresentacao
% ------------------------------------------------------------------------
% ------------------------------------------------------------------------

\documentclass[
	% -- opções da classe memoir --
	12pt,				% tamanho da fonte
	openright,			% capítulos começam em pág ímpar (insere página vazia caso preciso)
	oneside,			% para impressão em verso e anverso. Oposto a oneside
	a4paper,			% tamanho do papel. 
	% -- opções da classe abntex2 --
	chapter=TITLE,		% títulos de capítulos convertidos em letras maiúsculas
	section=TITLE,		% títulos de seções convertidos em letras maiúsculas
	%subsection=TITLE,	% títulos de subseções convertidos em letras maiúsculas
	%subsubsection=TITLE,% títulos de subsubseções convertidos em letras maiúsculas
	% -- opções do pacote babel --
	english,			% idioma adicional para hifenização
	french,				% idioma adicional para hifenização
	spanish,			% idioma adicional para hifenização
	brazil				% o último idioma é o principal do documento
	]{abntex2}

% ---
% PACOTES
% ---

% ---
% Pacotes fundamentais 
% ---
%\usepackage{lmodern}			% Usa a fonte Latin Modern			
\usepackage{times}			% Usa a fonte Times
\usepackage[T1]{fontenc}		% Selecao de codigos de fonte.
\usepackage[utf8]{inputenc}		% Codificacao do documento (conversão automática dos acentos)
\usepackage{lastpage}			% Usado pela Ficha catalográfica
\usepackage{indentfirst}		% Indenta o primeiro parágrafo de cada seção.
\usepackage{color}			% Controle das cores
\usepackage{graphicx}			% Inclusão de gráficos
\usepackage{microtype} 			% para melhorias de justificação
\usepackage{listings}			% para inserir código fonte

% ---
% Pacotes de citações
% ---
\usepackage[brazilian,hyperpageref]{backref}	 % Paginas com as citações na bibl
\usepackage[alf]{abntex2cite}	% Citações padrão ABNT
\usepackage{titlesec}

% --- 
% CONFIGURAÇÕES DE PACOTES
% --- 

% altera o espacamento depois do número de cada secao, subsecao, etc
\titleformat{\section}
  {\normalfont\normalsize}{}{0pt}{\thesection\space}
\titleformat{\subsection}
  {\normalfont\normalsize\bfseries}{}{0pt}{\thesubsection\space}
\titleformat{\subsubsection}
  {\normalfont\normalsize}{}{0pt}{\thesubsubsection\space}
\titleformat{\paragraph}
  {\normalfont\normalsize\itshape}{}{0pt}{\theparagraph\space}

% ---
% Configurações do pacote backref
% Usado sem a opção hyperpageref de backref
\renewcommand{\backrefpagesname}{Citado na(s) página(s):~}
% Texto padrão antes do número das páginas
\renewcommand{\backref}{}
% Define os textos da citação
\renewcommand*{\backrefalt}[4]{
	\ifcase #1 %
		Nenhuma citação no texto.%
	\or
		Citado na página #2.%
	\else
		Citado #1 vezes nas páginas #2.%
	\fi}%
% ---

% ---
% **************************************************
% Informações que devem ser alteradas
% **************************************************
\titulo{\uppercase{TÍTULO do trabalho \\ desenvolvido}}
\autor{\uppercase{Nome do Aluno}}
\orientador{Paulo Manoel Mafra}
\orientadortcc{Prof. \imprimirorientador, Dr. (SENAI/SC)}
\coordenador{Prof. Bobiquins Estevão de Mello, Me. (SENAI/SC)}
\coordenadortcc{Profa. Jaqueline Voltolini de Almeida, Me. (SENAI/SC)}
\examinador{Prof. Fulado de tal, Me. (SENAI/SC)}
\preambulo{Trabalho de Conclusão de Curso apresentado à Banca Examinadora do Curso Superior de Tecnologia em Redes de Computadores da Faculdade de Tecnologia do SENAI Florianópolis como requisito parcial para obtenção do Grau de Tecnólogo em Redes de Computadores.}
\proforientador{Professor Orientador: \imprimirorientador.}
\datadefesa{\uppercase{20 de dezembro de 2013}}
\local{Florianópolis/SC}
\data{2013}
% **************************************************

\instituicao{%
  SERVIÇO NACIONAL DE APRENDIZAGEM INDUSTRIAL 
  \par
  FACULDADE DE TECNOLOGIA SENAI/SC FLORIANÓPOLIS
  \par
  CURSO SUPERIOR DE TECNOLOGIA EM REDES DE COMPUTADORES}
\tipotrabalho{Trabalho de Conclusão de Curso}

% alterando o aspecto da cor azul
\definecolor{blue}{RGB}{41,5,195}

% informações do PDF
\makeatletter
\hypersetup{
     	%pagebackref=true,
		pdftitle={\@title}, 
		pdfauthor={\@author},
    	pdfsubject={\imprimirpreambulo},
	    pdfcreator={LaTeX with abnTeX2},
		pdfkeywords={abnt}{latex}{abntex}{abntex2}{trabalho acadêmico}, 
		colorlinks=true,      	% false: boxed links; true: colored links
		linkcolor=black,	% color of internal links
		citecolor=black,        % color of links to bibliography
		filecolor=magenta,      % color of file links
		urlcolor=black,
		bookmarksdepth=4
}
\makeatother
% --- 

% --- 
% Espaçamentos entre linhas e parágrafos 
% --- 

% O tamanho do parágrafo é dado por:
\setlength{\parindent}{1.3cm}

% Controle do espaçamento entre um parágrafo e outro:
\setlength{\parskip}{0.2cm}  % tente também \onelineskip

%\titlespacing\section{0pt}{12pt plus 4pt minus 2pt}{-6pt plus 2pt minus 2pt}

% ---
% compila o indice
% ---
\makeindex
% ---

% ----
% Início do documento
% ----
\begin{document}

% Retira espaço extra obsoleto entre as frases.
\frenchspacing 

% ----------------------------------------------------------
% ELEMENTOS PRÉ-TEXTUAIS
% ----------------------------------------------------------

\imprimircapa
\imprimirfolhaderosto*

\begin{folhadeaprovacao}

  \begin{center}
    {\ABNTEXchapterfont\bfseries\normalsize\imprimirautor}

    \vspace*{\fill}\vspace*{\fill}
    \begin{center}
      \ABNTEXchapterfont\bfseries\normalsize\imprimirtitulo
    \end{center}
    \vspace*{\fill}
        \imprimirpreambulo
    \vspace*{\fill}
        
   APROVADA PELA {\bfseries{COMISSÃO EXAMINADORA}}
   \par
   EM FLORIANÓPOLIS, \bfseries{\imprimirdatadefesa}
   \end{center}

   \assinatura{\imprimircoordenador \\ Coordenador do Curso} 
   \assinatura{\imprimircoordenadortcc \\ Coordenador de TCC} 
   \assinatura{\imprimirorientadortcc \\ Orientador} 
   \assinatura{\imprimirexaminador \\ Examinador}
   \begin{center}
    \vspace*{1cm}
  \end{center}
\end{folhadeaprovacao}

% Arquivos que devem ser alterados
% ====================================================================
% Dedicatória 
% ====================================================================
\begin{dedicatoria}
Dedico este trabalho ao meu avô Huri Gomes Mendonça, \textit{in memoriam}, que sempre me apoiou nos estudos mesmo sem entender o contexto do curso.

\end{dedicatoria}

% ====================================================================
% Agradecimentos 
% ====================================================================
\begin{agradecimentos}
Agradeço à minha família, minha noiva, meus amigos e colegas de trabalho pela ajuda e pelo apoio de sempre.
\end{agradecimentos}

% ====================================================================
% Epigrafe 
% ====================================================================
\begin{epigrafe}
    \vspace*{\fill}
	\begin{flushright}
		\textit{``Talk is cheap. Show me the code''} \\
		(LINUS TORVALDS)
	\end{flushright}
\end{epigrafe}


% ====================================================================
% Resumo 
% ====================================================================

\noindent
SOBRENOME, Nome. \textbf{Título do trabalho.}
Florianópolis, 2013. \pageref{nropaginas}f. Trabalho de Conclusão de Curso Superior de Tecnologia em
Redes de Computadores - Curso Redes de Computadores. Faculdade de Tecnologia do
SENAI, Florianópolis, 2013.

\vspace{1cm}
\setlength{\absparsep}{18pt} % ajusta o espaçamento dos parágrafos do resumo
\begin{resumo}
 Segundo a NBR6028:2003, o resumo deve ressaltar o
 objetivo, o método, os resultados e as conclusões do documento. A ordem e a extensão
 destes itens dependem do tipo de resumo (informativo ou indicativo) e do
 tratamento que cada item recebe no documento original. O resumo deve ser
 precedido da referência do documento, com exceção do resumo inserido no
 próprio documento. (\ldots) As palavras-chave devem figurar logo abaixo do
 resumo, antecedidas da expressão Palavras-chave:, separadas entre si por
 ponto e finalizadas também por ponto.

 \textbf{Palavras-chave}: Latex. Abntex. Editoração de texto.
\end{resumo}

% ====================================================================
% Abstract 
% ====================================================================
\noindent
SOBRENOME, Nome. \textbf{Título do trabalho.}
Florianópolis, 2013. 89f. Trabalho de Conclusão de Curso Superior de Tecnologia em
Redes de Computadores - Curso Redes de Computadores. Faculdade de Tecnologia do
SENAI, Florianópolis, 2013.

\vspace{1cm}
\begin{resumo}[\textbf{ABSTRACT}]
 \begin{otherlanguage*}{english}
   This is the english abstract.

   \vspace{\onelineskip}
 
   \noindent 
   \textbf{Key-words}: Latex. Abntex. Text editoration.
 \end{otherlanguage*}
\end{resumo}



% lista de figuras
\pdfbookmark[0]{\listfigurename}{lof}
\listoffigures*
\clearpage

% inserir lista de tabelas
\pdfbookmark[0]{\listtablename}{lot}
\listoftables*
\clearpage

% Arquivos que devem ser alterados
% ====================================================================
% Siglas 
% ====================================================================

\begin{siglas}
  \item[CGroups] Control Groups
  \item[LXC] Linux Container
  \item[VPS] Servidores privados virtuais
  \item[TI] Tecnologia da Informação
\end{siglas}


\include{alterar/simbolos}

% inserir o sumario
\pdfbookmark[0]{\contentsname}{toc}
\tableofcontents*
\clearpage

% ----------------------------------------------------------
% ELEMENTOS TEXTUAIS
% ----------------------------------------------------------
\textual

% PARTE - preparação da pesquisa
% ----------------------------------------------------------
%\part{Preparação da pesquisa}


% Informações que devem ser alteradas
% **************************************************
% ---
% Introdução
% ---
\chapter{INTRODUÇÃO}

fjsdlfjsjfdslfjfdsvfsdfsdfasdfsdfsdfasfasdfasfsd fsljsldfjsdlf sdlfjsdlkfjsdljflkf sdlkfjdslfksdfjs dlfjsdfldjsfklsfj.

Atualmenteas asfslfjslfsdfsd
sfsdlfjdskfjdskfjsdljfldfjldsfjs
sdfjdwjfdkfjdwkfjdsf
sfsdf sfsdfsdf

\textsf{abntex2} e do pacote \textsf{abntex2cite}.

\section{JUSTIFICATIVA}

De-se justificar a escolha do tema, a finalidade, relevância e foco do assunto.

\section{OBJETIVOS}

Nesta seção são apresentados os objetivos do presente trabalho.

\subsection{Objetivo geral}

Estudar o funcionamento da ferramenta \LaTeX.

\subsection{Objetivos específicos}

Aqui são definidos os objetivos específicos do trabalho.

\section{METODOLOGIA}

Dizer qual metodologia de trabalho será usada.

\section{ESTRUTURA DO TRABALHO}

Explicar como o trabalho está estruturado.

% ---
% Capitulo de revisão de literatura
% ---
\chapter{REVISÃO DA LITERATURA}\label{cap-revisao}

Neste capítulo são apresentados trabalhos relacionados que sejam relevantes para a pesquisa. Isso é um exemplo de citação \citeonline[p.10]{NBR10520:2002}. Os dados de cada citação devem ser inseridos no arquivo \textit{alterar/referencias.bib}.

A seguir é apresentado uma citação direta com mais de três linhas:

\begin{citacao}
As citações diretas, no texto, com mais de três linhas, devem ser destacadas com recuo de 4 cm da margem esquerda, com letra menor que a do texto utilizado e sem as aspas. No caso de documentos datilografados, deve-se observar apenas o recuo \cite[p. 10]{NBR10520:2002}.
\end{citacao}

Na sequência é apresentado um exemplo simples de como escrever o TCC no \LaTeX. Primeiramente os campos no início desse arquivo devem ser preenchidos (título do trabalho, nome do autor, etc). Na sequência, o texto dos capítulos deve ser inserido, conforme esse exemplo. O arquivo de referências bibliográficas também deve ser gerado, conforme exemplo citado acima. Para ``compilar'' esse modelo e gerar o arquivo no formato ``.pdf'', no GNU/Linux, os seguintes comandos devem ser executados: \textbf{pdflatex modelo-tcc-senai.tex}, \textbf{bibtex modelo-tcc-senai} e mais duas vezes \textbf{pdflatex modelo-tcc-senai.tex}. 

Veja um exemplo de nota de rodapé\footnote{Isso é uma nota de rodapé.}.

\begin{citacao}
A Internet é composta por um misto de diversas estruturas de redes administradas 
de forma descentralizada, dificultando a migração para o IPv6. Além disso, mesmo que 
a administração fosse centralizada, muitas estruturas não suportam a implantação do 
protocolo IPv6 \cite{martini2003analise}
\end{citacao}

Podemos também inserir tabelas da seguinte forma. A Tabela \ref{tab-exemplo} mostra um exemplo de tabela.

\begin{table}[!h]
\center\scriptsize
\caption{\textbf{Tabela exemplo}}
\begin{tabular}{|l|l|l|l|l|}
\hline
fsdfdfsdfsadf & dsfsdf & sfsd   & sdfsdfdd & sdfsdfsdsdfsdfsdfsdfsdfsdfsdfsdfsdf \\ \hline
sfsdfsdfsd    & sfsdds & fsdsds & dsfsds   & sfsdfsd                             \\ \hline
              &        &        &          &                                     \\ \hline
              &        &        &          &                                     \\ \hline
\end{tabular}
\fonte{do autor}
\end{table}

\begin{table}[!h]
\center\scriptsize
\caption{\textbf{Tabela exemplo}}
\begin{tabular}{|l|c|c|} \hline \label{tab-exemplo}
\textbf{ Característica}	& \textbf{Valor} 	& \textbf{ Comportamento} \\ \hline
Característica A 		& 8			& Bom \\ \hline
Característica B		& 10			&Ótimo \\ \hline
\textbf{Situação}		& \multicolumn{2}{c|}{\textit{Aprovado!}} \\ \hline
\end{tabular}
\fonte{do autor}
\end{table}



Na sequência inserimos uma figura para exemplo. A Figura \ref{fig-exemplo} é um exemplo de como devem ser inseridas as figuras no documento.

\begin{table}[!htb]
\center\scriptsize
\caption{\textbf{Outra tabela}}
\begin{tabular}{|l|l|l|l|l|l|}
\hline \label{outra-tab}
Coluna 01  & Coluna 02  & Coluna 03  & Coluna 04  & Coluna 05  & Coluna 06 \\ \hline
zzzzzzzzzz & zzzzzzzzzz & zzzzzzzzzz & zzzzzzzzzz & zzzzzzzzzz &           \\ \hline
           &            &            &            &            &           \\ \hline
           &            &            &            &            &           \\ \hline
\end{tabular}
\fonte{Do Autor(2014)}
\end{table}


\begin{figure}[htb]
  \begin{center}
    \caption{\textbf{Figura de exemplo}}
    \label{fig-exemplo}
    \includegraphics [scale=0.6]{logo-senai.jpg}
    \fonte{\cite{nemeth2004manual}}
    \label{fig-exemplo}
  \end{center}
\end{figure}



\section{LINUX}

Exemplo de uma nova seção.
Conforme \citeonline{nemeth2004manual}:

\begin{citacao}
Se um processo ficar frenético e levar a média de carga do sistema para 65, talvez seja necessário utilizar nice para iniciar um shell de alta prioridade antes de executar comandos para investigar o problema. Caso contrário, pode ser difícil executar até mesmo comandos simples.\cite[p.44]{nemeth2004manual}
\end{citacao}

\subsection{Nova Subseção}

\citeonline[tradução nossa]{beser2014dynamic} pressupõe que "Certas formas de realização permitir a prestação de serviços de acesso de banda larga sem fio que parecem ter as mesmas características que os serviços de ADSL ou cabo de rede com fio."

\subsubsection{Nova Subsubseção}

Exemplo de uma nova subsubseção.

\section{SOA}

Conforme proposto por \citeonline[p.15, tradução nossa]{rotem2012soa}, "..........".

\begin{citacao}
faço a minha citação de mais de três linhas, mais de três linhas, mais de três linhas, mais de três linhas.\cite{josuttis2007soa}
\end{citacao}

\citeonline{artigoqualquer2014} ...




\paragraph{Nova Subsubsubseção}

Exemplo de uma nova subsubsubseção.

% ---
% Procedimentos metodológicos
% ---
\chapter{PROCEDIMENTOS METODOLÓGICOS}

Descrevem-se, neste capítulo, os procedimentos metodológicos que nortearam a pesquisa.

% ---
% Resultados
% ---
\chapter{RESULTADOS E DISCUSSÕES}

Neste capítulo são apresentados os resultados da pesquisa descrita no capítulo \ref{cap-revisao}.

% ---
% Conclusão
% ---
\chapter{CONCLUSÃO}

As conclusão do trabalho são apresentadas aqui.

% **************************************************

% ----------------------------------------------------------
% ELEMENTOS PÓS-TEXTUAIS
% ----------------------------------------------------------
% \postextual

% ----------------------------------------------------------
% Referências bibliográficas
% Arquivos que devem ser alterados
\bibliography{alterar/referencias}

% ----------------------------------------------------------
% Glossário
% ----------------------------------------------------------
%
% Consulte o manual da classe abntex2 para orientações sobre o glossário.
%
%\glossary

% ----------------------------------------------------------
% Apêndices
% ----------------------------------------------------------

% Informações que devem ser alteradas
% **************************************************
% ---
% Inicia os apêndices
% ---
\begin{apendicesenv}

\chapter{Código fonte}
Código de minha autoria. O apêndice é opcional ao TCC e deve ser elaborado pelo próprio autor. Destina-se a complementar as ideias, sem prejuízo do tema do trabalho. Segue um exemplo:

\scriptsize
\begin{lstlisting}
#include <stdio.h>

int main() {
  printf("Ola mundo !\n");
  return 0;
}
\end{lstlisting}

\end{apendicesenv}

% ----------------------------------------------------------
% Anexos
% ----------------------------------------------------------
\begin{anexosenv}

\chapter{Pesquisa IBGE}
O anexo é opcional ao TCC e são informações não elaboradas pelo próprio autor, mas que tem como objetivo complementar as ideias, sem prejuízo do tema do relatório.

\end{anexosenv}

% Etiqueta para auxiliar contagem do numero de paginas do texto e dos elementos pos-textuais
\label{nropaginas}

% **************************************************

%---------------------------------------------------------------------
% INDICE REMISSIVO
%---------------------------------------------------------------------
\printindex

\end{document}
